\documentclass[UTF8]{ctexart}

\usepackage{geometry}
\usepackage{listings}
\usepackage{natbib}
\usepackage{graphicx}
\usepackage{subcaption}
\usepackage{algorithm}
\usepackage{amsmath}
\usepackage[noend]{algpseudocode}
\usepackage{lmodern}
\usepackage{url}
\usepackage{verbatim}
\usepackage{multirow}

\lstset{frame=tb,
  aboveskip=3mm,
  belowskip=3mm,
  showstringspaces=false,
  columns=flexible,
  basicstyle={\small\ttfamily},
  numbers=none,
  numberstyle=\tiny\color{gray},
  breaklines=true,
  breakatwhitespace=true,
  tabsize=3
}

\geometry{left=4cm,right=4cm,top=3cm,bottom=3cm}

\title{“软件项目中的库依赖变化”研究计划}
\author{何昊}

\begin{document}

\maketitle

\section{研究动机}

近年来,开源软件库的数量爆发性增长,并且工业界使用的开源软件也越来越多。
现在,开发者可以使用例如NPM这样的包管理器来管理一个项目所使用的开源库。
开源库也有可能使用其他库,这些依赖关系合在一起,形成了软件供应链。
现在,一个软件项目很可能会直接和间接依赖大量的库。
保证这些库组成的软件供应链的安全性是工业界关心的重大课题。
近年来,学术界也出现了大量对软件供应链的研究,例如库与库之间的依赖关系和依赖网络的研究\cite{2017MSR-Kikas-Structure, 2019MSR-Valero-DiversityMaven};开发者是否升级一个依赖库\cite{2018EMSE-Kula-Do, 2018ICSME-Zapata-Towards};Semantic Versioning \cite{2019MSR-Dietrich-DependencyVersioning, 2019TSE-Decan-Semantic}。

众所周知,软件项目会快速地动态变化。
随着软件项目的不断演化,它所对应的供应链也很可能会变化。
在开发过程中,开发者可能会新添加一个库,可能会移除一个库,也可能将某个库升级到了更新的版本。
这个原因可能是由于项目本身,也可能是依赖库的原因。
已有研究主要聚焦与开发者是否升级一个依赖,尚没有一个全面的对项目演化过程中软件供应链的变化的研究。
我们并不知道,项目开发过程中,库依赖会发生什么样的变化,以及为什么会发生这样的变化。

我认为,调研这样的依赖变化(Dependency Variations)很可能是有用的,能够发现一些对学术界、工业界和开发者都有用的结论,并更好地辅助开发过程。
而且,这个研究还可以从如下角度来辅助库迁移的研究。
已有的库迁移研究使用项目中移除和添加的依赖项来判断是否存在库迁移。
但是,我认为这个方法是有一点问题的。
我们不仅不知道开发者在开发过程中为什么会删除一项库依赖,也不知道开发者在开发活动中为什么会添加一项依赖。
如果有了这方面的经验性证据,则可以设计更加准确的库迁移检测方法。

\section{研究问题}

\begin{itemize}
    \item RQ1: 在项目开发过程中,项目依赖会发生哪些变化?
    \item RQ2: 这些变化发生的频率有多高?%大规模的情况下,还可以研究发生变化的项目比例
	\item RQ3: 不同的项目在开发过程中的依赖变化有何不同?
	\item RQ4: 有没有在不同项目之间频繁出现的依赖变化?这些频繁出现的依赖变化有什么特征?
    \item RQ5: 为什么会发生这些依赖变化?
	\item RQ6:依赖变化具有什么样的模式,例如删除库和添加库的时间间隔会是多少,是否一定会立即移除已经不使用的库。是否跟库和项目类型有关。
	\item RQ7:依赖变化后的代码活动上有什么模式,是否会集中一个时段做迁移,或是在较长的一段时间内完成。单个commit来看的话,是否一个commit会涉及一段完整功能的迁移,或是一个功能会分为几个commit来做。在不同类型的库或项目上有没有区别。
\end{itemize}

\section{研究方法}

\subsection{数据集}

初步研究:在GitHub上选取5个大型项目对其开发历史做详细分析

扩展研究:GitHub上Star数最多并使用了包管理器的Java和JavaScript项目各500个(便于快速迭代和探索结果)

最终研究:Libraries.io或者World of Code \cite{2019MSR-Ma-WorldOfCode}上的项目(用于构建最终的实验数据)

\subsection{使用的符号和定义}

$P$表示数据集中包含的软件项目的集合,$L$表示数据集中包含的所有库的集合。
对于每个项目$p \in P$,如果项目$p$有$n$个不同的commit,$\langle p, c_i \rangle, i \in [0, n - 1)$表示这个项目的第$i$个commit。
对于每个库$l \in L$, 类似地,我们用$\langle l, v_i \rangle$来表示库$l$的版本$v_i$。

如果项目$p$的$c_i$中依赖了$m$个库,定义
\begin{equation}
dep_p(c_i)=\{\langle l_1, v_1 \rangle,  \langle l_2, v_2 \rangle..., \langle l_m, v_m \rangle \}
\end{equation}
为项目$p$的$c_i$中所依赖的所有库及其对应版本。

\subsection{RQ1: 在项目开发过程中,项目依赖会发生哪些变化?}

给定$dep_p(c_i)$和$dep_p(c_j)$,$c_j$是$c_i$的后一个commit,定义以下五种依赖变化
\begin{align}
add(c_i, c_j) &= \{\langle l, v \rangle |l \notin dep_p(c_i) \wedge l \in dep_p(c_j)\} \\
del(c_i, c_j) &= \{\langle l, v \rangle |l \in dep_p(c_i) \wedge l \notin dep_p(c_j)\} \\
verchg(c_i, c_j) &= \{\langle l, v_m, v_n \rangle | \langle l, v_m \rangle \in dep_p(c_i) \wedge \langle l, v_n \rangle \notin dep_p(c_j)\}\\
update(c_i, c_j) &= \{\langle l, v_m, v_n \rangle | \langle l, v_m, v_n \rangle \in verchg(c_i, c_j) \wedge v_n < v_m \} \\
rollback(c_i, c_j) &= \{\langle l, v_m, v_n \rangle | \langle l, v_m, v_n \rangle \in verchg(c_i, c_j) \wedge v_n > v_m \}
\end{align}
从前往后依次表示项目在commit之间可能发生的添加一个库,删除一个库,改变一个库的版本,升级一个库和降级一个库五种变化。由于开发者可能会采用一种宽松的依赖版本定义策略\cite{2019MSR-Dietrich-DependencyVersioning},因此$verchg(c_i, c_j) \neq update(c_i, c_j) \cup rollback(c_i, c_j)$。

项目的所有commit构成一个有向图$G=\langle C, E \rangle$。对于每个$\langle c_i, c_j \rangle \in E$,可以分别计算出以上所有符号所表示的集合,以此得到不同的依赖变化类型及其所对应的内容。将这些数据综合在一起,就是项目开发过程中发生的所有依赖变化数据。

基于这些数据,我们可以分析
\begin{enumerate}
	\item 各个依赖变化类型的占比。
	\item 什么样的升级、添加、删除行为最为频繁。
	\item 有没有什么独特的特征,例如添加一个库和删除一个库同时出现表示库迁移。
\end{enumerate}

\subsection{RQ2: 这些变化发生的频率有多高?}

分析每种变化平均每多少个commit会发生一次

\subsection{RQ3: 不同的项目在开发过程中的依赖变化有何不同?}

分析依赖变化与其他项目指标的关系,例如
\begin{enumerate}
	\item 什么样的项目,或者处于什么阶段的项目会发生很多的依赖变化?
	\item 是否越活跃的项目会发生更多的依赖变化?在什么方面活跃的项目更倾向于发生依赖变化?
	\item 上面的结论对于添加、删除和升级三种不同的依赖变化分别有什么不同?
\end{enumerate}

\subsection{RQ4: 有没有在不同项目之间频繁出现的依赖变化?}

将以上定义的各种依赖变化模式综合到一起,在全部项目的数据上查看有没有频繁出现的变化,再观察频繁出现的变化是否有某种与时间有关的特征。例如,哪些相同的变化会集中在某个时间点出现;又例如,是否会上游先变化带动下游变化,等等。

\subsection{RQ5: 为什么会发生这些依赖变化?}

方法1:分析前后的commit message来推测原因

方法2:向这个commit的作者发送问卷

\section{期待的结果}

也许我们可以发现,整体而言,依赖变化在开发过程中不是很常见,一开始可能变化会很多,当项目稳定之后变化会越来越少。在所有的依赖变化中,由于新需求而添加一个依赖库是很常见的,但是删除一个库不是很常见,可能会出于一些复杂的原因,例如库迁移。也许删除库99\%都是与库迁移有关,也许还有一些其他的复杂原因。而且,可能会存在明明项目已经不使用了某个库但是却没有将其从依赖中删除的情况,使得一个项目会引入无谓的依赖。也许我们可以从中总结出一个可靠的挖掘库迁移的方法,或者发现通过推测依赖变化发现库迁移是不可取的。等等。

\section{初步结果}

1. 平均每10个Commit会发生一次依赖变化

2. 各种依赖变化的发生频率:版本变化 > 添加 > 删除

3. 实际的库迁移的例子很少,具体有


\bibliographystyle{plain}
\bibliography{references}
\end{document}